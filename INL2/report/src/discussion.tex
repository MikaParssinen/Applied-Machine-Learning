\section{Discussion}


\subsection{Comparing the models}
Based on the results, we can see that DBSCAN does not provide useful clusters. 
It either assumes that approximately 80\% of the data is noise and groups it into a single cluster, or it creates too many clusters, resulting in overly niche and small groups that offer no meaningful insights into customer behavior.
\par
Comparing this to the results of the Hierarchical Clustering, we obtain groups of customers with distinct behaviors, which provides valuable insights for each group. Additionally, Hierarchical Clustering allows us to visualize the data more effectively through a dendrogram, which illustrates how clusters are formed and combined.
This is might be useful for the economics team, as they may want to adjust the number of clusters, based on how each cluster is formed. 
The flexibility offered by this method is especially important, as it does not require us to predefine the number of clusters, allowing for a more fine approach in identifying customer segments.


\subsection{Cluster Analysis for Targeted Marketing}
The results from Hierarchical clustering reveals 5 distinct clusters, 
each of the clusters shows how different groups of credit card users behaves. 
In other words we now have a deep insight of our customers habits and can therefore tailor our products for each of the groups.
For instance, the results above describes \textbf{Cluster 4} as: Customers with low purchases, high cash advances and low full payment. 
This cluster has the size of 1790 costumers, which means that approximately 22\% costumers use their card almost only for cash advances. 
\par
Great! We can now target marketing these costumers and recommend using our other credit cards which offers better benefits for using the card for cash advances.

