\section{Normalization and outlier removal}

To get better accuracy in our classifiers, normalization and removal of outliers is sometimes needed. This section explains the choice of normalizing or not for each model, as well as how we deal with outliers. We also explain which normalization technique we use for the models that we do normalize. 

\subsection{Support vector machine (SVM)}

\subsection{Naive bayes classifiers}

The naive bayes classifiers assume that each feature is independent of eachother. This means that the model will work best with the data not being normalized. We would gain no more information by normalizing the data and a consequence of normalizing could be that we lose information.
\par
Since we assume that the data comes from a normal distribution, we can remove outliers from the bottom and the top of the distribution. The percentiles chosen are $0.01$ and $0.99$. Removing these outliers improves the accuracy of the model slightly, further going to the $0.05$- and $0.95$-percentiles reduces the accuracy significantly.

\subsection{Random forest}