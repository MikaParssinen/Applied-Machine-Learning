\section{Sampling data}

To solve our problem of predicting the preferred position of a football player, we chose to use a subset of columns in the complete dataset. The set of features we took from this dataset was every column that describes a specific ability of a player, such as acceleration. To avoid attribute noise we did not include goalkeeper attributes. Since we exclude the goalkeeper position from our classification problem we won't lose any information by doing so. 
\par
Each player has a set of ``preferred positions'' specified in one of the columns in the dataset, this will be the labels we use in training and prediction. We will exclude all rows where ``Preferred Positions'' contains ``GK'' as in goalkeeper. With the player attributes and the ``Preferred Positions'' column, we can construct our Abstract Base Table.
\par
There is one issue with the ``Preferred Positions'' column however, we would like to predict just one position for each player, but the column specifies multiple positions for some players. To deal with this issue we have split each row containing multiple positions into one row for each position. How we do this is illustrated in \autoref{fig:split_rows} below.
\begin{figure}[!ht]
    \centering
    \begin{tikzpicture}[shorten >=0.5pt, node distance=1cm, auto]
        \node (first) {
            \begin{tabular}{|c|c|c|c|}
                \hline
                \rowcolor{gray!50}
                Acceleration & Aggression & $\cdots$ & Preferred Positions \\ \hline
                $89$ & $63$ & $\cdots$ & ST LW \\ \hline
            \end{tabular}
        };
        \node (second) [below=of first] {
            \begin{tabular}{|c|c|c|c|}
                \hline
                \rowcolor{gray!50}
                Acceleration & Aggression & $\cdots$ & Preferred Positions \\ \hline
                $89$ & $63$ & $\cdots$ & ST \\ \hline
                $89$ & $63$ & $\cdots$ & LW \\ \hline
            \end{tabular}
        };
        \path[->, >={Stealth[length=3.5mm, width=2.5mm]}] (first) edge (second);
    \end{tikzpicture}
    \caption{One player with two preferred positions split into two rows}\label{fig:split_rows}
\end{figure}
\par
This split does create another problem when it comes to prediction. The true label of each player is still a set of labels rather than just one label. Consequence of this is that when predicting a label for a player with multiple preferred positions, we will have multiple identical feature vectors with different labels. This means that for a player with $4$ preferred positions, we will label the $4$ feature vectors correct, at most, $1$ out of $4$ times.
\par
To deal with this issue, we chose to just split the training data and keep the test data as is. This means that for a prediction to be accurate, the predicted label needs to be one of the preferred positions. For players with multiple preferred positions, there are multiple correct answers. 