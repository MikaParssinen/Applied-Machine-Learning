\section{Discussion}

By used methods of tuning hyperparameters, normalizing data, cross-validation and feature engineering, it can be concluded that from our experiments that Support Vector Machines offer the best models for our problem. 

NOTES:
\begin{itemize}
    \item The results show that the pattern in the feature vectors for the different positions aren't distinct enough to create a predictive model that can accurately predict the position of a player. Especially when it comes to the difference between a left-sided player and a right-sided player.
    \item Synthetic data, the data is based on a mix of opinions and real-world data. There is no concrete way to measure the dribbling skills of a football player. Also with game balancing.
    \item Missing data, if a player is left-footed or right-footed would play a significant role of which side of the field the player prefers to play on.
    \item Naive bayes is fast to learn and predict, but the model is not as accurate as random forest. The other models are however, much more computationally expensive. 
    \item Result is limited by computational power and time.
    \item The provided dataset is only for male fotball players.

\end{itemize}