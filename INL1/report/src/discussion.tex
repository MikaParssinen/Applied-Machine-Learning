\section{Discussion}

By used methods of tuning hyperparameters, normalizing data, cross-validation and feature engineering, it can be concluded that from our experiments that Support Vector Machines offer the best models for our problem. 

Even by allowing for labelling a subset of the preferred positions, we do not get the accuracy above $80\%$. By this, we conclude that the feature vectors are not distinct enough for each each position to create a predictive model that can accurately predict the preferred position of a player. That each player have multiple preferred positions also potentially creates noise, human inspection of the data indicates no clear pattern of why a player prefers multiple positions.
\par
Another problem with the dataset is that there are no attributes that indicates a reason for why a player prefers the left, center or right side of the field. Having a feature that indicates if a player is left-footed or right-footed would play a significant role of which side of the field the player prefers to play on. This is another limitation of the dataset that we have to work with. 

NOTES:
\begin{itemize}
    \item Synthetic data, the data is based on a mix of opinions and real-world data. There is no concrete way to measure the dribbling skills of a football player. Also with game balancing.
    \item Naive bayes is fast to learn and predict, but the model is not as accurate as random forest. The other models are however, much more computationally expensive. 
    \item Result is limited by computational power and time.
    \item The provided dataset is only for male fotball players.
\end{itemize}

