\section{Discussion}

By tuning hyperparameters, normalizing data, cross-validation and feature engineering, it can be concluded that from our experiments that Support vector machine (SVM) offer the best models for our problem. The naive bayes classifier is the fastest to train and predict, but the model is not as accurate as the Random forest (RF) and SVM. 
\par
Due to our limitation in computational power we could not scale the RF and SVM as we desire. 
For instance, training the SVM model became extremely time-consuming because of the large number of calculations involved, especially when working with larger datasets. 
We tried to use SMOTE to balance the dataset by increase the underrepresented classes, and potentially increase the accuracy. But this significantly slowed down computations and increased runtime so we never saw any increase in accuracy.
\par
Similarly, the RF model's n\_estimators hyperparameter, which determines the number of trees in the forest, could not be set to high values. 
Although a larger number of estimators often improves performance, the computational load was too great for higher values, so we had to settle for settings our hardware could handle.
\par
Despite these efforts, even after labeling subset of the preferred positions, we do not get the accuracy above $80\%$. By this, we conclude that the feature vectors are not distinct enough for each each position to create a predictive model that can accurately predict the preferred position of a player. That each player have multiple preferred positions also potentially creates noise, human inspection of the data indicates no clear pattern of why a player prefers multiple positions.
\par
Another problem with the dataset is that there are no attributes that indicates a reason for why a player prefers the left, center or right side of the field. Having a feature that indicates if a player is left-footed or right-footed would play a significant role of which side of the field the player prefers to play on. This is another limitation of the dataset that we have to work with. 
\par 
Additionally, the dataset only includes male football players, which restricts the scope of our predictions.
This limitation means that predictions based on the data may not be applicable to female football players or mixed datasets, narrowing our problem statement to predicting only the positions of male football players.
\par
We must also remind ourselves that this data is synthetic, meaning it does not represent real-world data from actual football players.  
Instead, the data reflects in-game data from a football game, how each statistic is decided is a mystery, simply they do not correspond to real player statistics.
\par
Since the data comes from a game, balancing player stats is crucial to ensure no unfair advantage is given to players who own a particular in-game football player.
However, this balancing introduces noise into the data.
This is especially relevant to our problem, where we aim to predict a football player’s position. In reality, what we are predicting is the preferred position of a FIFA in-game football player, not a real-world player.





