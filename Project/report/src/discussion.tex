\section{Discussion}

\subsection{Ethical consideration}

Our implementation of a machine learning model to classify bottles in a factory brings a few important ethical considerations. 
While the technology offers great improvements in quality control and efficiency, it also presents challenges that need to be discussed and thought about.

First of is data security and privacy. 
The system processes large amounts of production data that could be valuable to competitors. 
Factories that uses our implementation is therefore recommended to implement a robust cybersecurity system to protect against industrial espionage and unauthorized access. 
This includes secure data storage, encrypted communications, and strict access controls for system modifications.

Additionally, our implementation leads to workforce changes. While some traditional quality control positions may be displaced, new roles emerge in system maintenance, monitoring, and data analysis. 
This transformation requires human resources, they should present retraining programs and use clear communication with affected employees.
To ensure employees gets the right and fair treatment.

Finally, accountability and responsibility of our implementation present another critical consideration. 
If defective products reach customers, clear frameworks must be establish whether responsibility lies with the system developers, factory management, or quality control staff. 
This requires transparent documentation of the system's decision-making process and regular reviews of its performance.

In conclusion, ethical implications of implementing this technology extend beyond the factory product line.
While automation can improve product quality and reduce waste, organizations must balance these benefits against their social responsibilities. 
This includes maintaining employment opportunities while adapting to technological advancement, ensuring transparent communication with stakeholders, and maintaining human oversight in critical decision-making processes.

\subsection{Limitations}

Our primary limitation is that we use \textbf{synthetic data} which means that the data is not real, it is synthetically made.
This could result in varying model performance between a controlled environment and a real-world setting.

Also the data we have is considered \textbf{small data}, which limits our model performance. 
Because, machine learning models generally perform better with larger datasets, and the limited data available may not capture all the variations and anomalies that occur in a real production environment.

Another limitation is the introduction of \textbf{fake noise} in the data.
While we have simulated noise in the data, real noise in a factory setting would include factors such as dust and dirt on the camera lens. 
This type of noise could significantly impact the model's accuracy and robustness.

Additionally, the data used lacks \textbf{all real anomalies}, meaning it does not fully represent all possible anomalies that could occur in a real-world scenario. 
This could lead to the model being less effective at detecting rare or unexpected defects.

Finally, the \textbf{contamination is clearly fabricated}.
This could result in the model learning to detect these fabricated patterns rather than genuine anomalies, reducing its effectiveness in a real production environment.

\subsection{Results}

\subsection{Future work}