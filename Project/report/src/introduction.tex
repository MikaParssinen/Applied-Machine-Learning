\section{Introduction}

In today’s fast-paced manufacturing environment, bottle factories are under constant pressure to produce pristine glass bottles in large volumes at high speeds, while not making a too big of an impact on the environment. The ever-growing demand for glass bottles \cite{ReturnableBottles} intensifies the challenges faced by these factories, requiring them to maintain efficiency without the cost of quality.  To manage external expectations, the factory must continuously monitor the production line, if they do that correctly they will be able to recognize defect bottles, such as broken and contaminated bottles. Therefore the factory will find out when a bottle is defect and might even get the answer, why the bottle is defect. Leading the factory to make better decisions for example recycle the broken bottles directly and clean the contaminated bottles to save both money and the environment.
\par
Traditionally, factories have a worker that stands by the production line, sometimes for only parts of the day or, at other times, for the entire day. The worker manually visually inspect the bottles which not only cost a lot of money in salary but also will bottlenecks the whole production of bottles. A human can simply not keep track of the hundreds of bottles that the factory produces, humans are slow and error prone to this kind of work, while not mentioning how boring the work must be. Therefore this solution dose not satisfy the external expectations.  A different monitor solution must exist.
\par
Monitoring with cameras, by incorporating automated monitor systems in the production line, such as cameras. 
With the camera, the output feed can be used as input data in a machine learning algorithm. 
These algorithms has been trained to identify with high accuracy defect bottles. 
They can do that by analyzing the images from the camera. Therefore, acquiring the features of the bottle and then make a prediction on the status of the bottle, if it is whole, broken or contaminated.
If a bottle is broken, we can identify the specific features indicating the breakage, which could help the factory understand the cause of the defect.
Helping the factory to improve the production so less or none bottles are broken. 
\par
The approach of monitoring the production line with cameras offers a bunch of environmental and economical benefits. Broken bottles can be sent of to recycling that means the factory will reduce waste and help the environment, contaminated bottles can be cleaned and reused, saving the the factory  money. This surely sound as the best solution.
\par
This report will show the implementation of machine learning algorithms, evaluate the effectiveness of identifying defect bottles and also discuss the results of each algorithm. Additionally, mention limitations, challenges and ethical considerations of these technologies.
