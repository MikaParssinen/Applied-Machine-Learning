\section{Introduction}

In todays standard, bottle factories are expected to run and produce bottles at high rates. 
While maintaining these high rates of production, the quality of the product should be excellent and the occurrence of production defects should be minimal.
To ensure excellent quality and high production rates, broken and contaminated bottles must be identified, sorted and separated as soon as possible. 
One way to do this is to monitor the production line, with cameras to identify whenever a bottle is either contaminated or broken. 
The identification is done by providing the camera feed as input to machine learning algorithms, which has been trained to identify broken and contaminated bottles.
With this addition to the production line, the company can choose to recycle the broken bottles and clean the contaminated bottles, reducing the factory environmental impact while saving money.

\par
This report will include implementation of machine learning algorithms, evaluation and discussion of results, limitations and ethical aspects. 