\section{Conclusion}

Our paper demonstrates the effectiveness of an autoencoder-based anomaly detection system combined with a CNN for classifying bottle defects, while still using a small dataset.
The anomaly detection procedure accurately identifies anomalies with minimal errors, and the CNN successfully classifies the defects. 
However, consider the limitations such as reliance on synthetic data, overrepresentation of a certain class, and the introduction of synthetic noise may affect the performance in the real world.
While the model performs well in a controlled setting, its generalizability to actual factory conditions remains unknown.

Future improvements include testing alternative architectures, enhancing data augmentation techniques, and deploying the model in a real production environment to collect authentic defect data. 
Ethical considerations such as data security, workforce impact, and accountability must also be addressed when implementing such systems in industry. 
Overall, this paper highlights the potential of deep learning for automated defect detection, but further validation by applications in the real world is necessary to ensure reliability and robustness.